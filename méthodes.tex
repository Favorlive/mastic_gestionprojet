\chapter{Méthodes de gestion de projet}
Afin de mener à bien le projet efficacement, il sera nécessaire de suivre des méthodes de gestion de projet.
\section{La Gouvernance du projet avec Prince2}
Dans un grande institution tel qu'une université, il est important d'avoir une gestion de projet proche d'une entreprise avec une maîtrise important des processus.

Prince2 permet d'avoir un contrôle sur la gestion du projet avec la constitution d'un comité de pilotage et d'un chef de projet. Nous pourrions très bien imaginer le comité de pilotage avec les responsables de l'Université en charge de la communications, des travaux, de l'entretien, de l'informatique, des représentants des facultés; bref, toutes personnes qui de près ou de loin seront impliqués dans l'utilisation des bornes. 

Le chef de projet ensuite exécutera le projet en accord avec les attentes et exigence du comité de pilotage. Il sera également chargé d'organiser le travail des différentes équipes qui participeront à la livraison du produit. Le comité de pilotage quant à lui assurera un rôle de validation tout au long du projet. 

Les besoins des différentes facultés ou des acteurs qui utiliseront la borne sont peut être très différent. Peut-être que l'une faculté préfère l'information vers les étudiants au détriment de l'aide à l'orientation des visiteurs par exemple. Dans la méthode Prince2, il faut pouvoir se mettre d'accord sur un produit qui pourra répondre aux besoins de tous les membres du comité de pilotage. Cette structure organisationnelle permet d'avoir un contrôle plus forte. 

\section{Le développement logiciel sur les bornes: Agile Scrum}
Pour les applications et les logiciels présent dans les bornes, nous allons opter pour une gestion de projet de type Agile. En effet, à ce stade du projet, nous nous situons dans l'étape Prince2 de livraison du projet. 

Nous avons besoin d'une personne qui puisse définir l'ensemble des besoins dans le logiciel (exemple: avoir une fonctionnalité de recherche des bâtiments, une possibilité de montrer l'itinéraire à l'utilisateur, d'afficher les correspondances de bus pour se rendre sur un autre campus, etc.) dans le product backlog. Un Scrum Master sera également présent pour faciliter la livraison et pouvoir veiller à la bonne application de la méthode. 

Le process par itération permettra d'avoir rapidement un produit utilisable que l'on soumettra éventuellement aux utilisateurs de tests. Ces utilisateurs vont pouvoir faire des retours afin d'améliorer le produit de sprint en sprint. 