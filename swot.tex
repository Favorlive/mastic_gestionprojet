\chapter{Analyse SWOT}
L’analyse SWOT (table \ref{tb_swot}) est un outil d’analyse stratégique qui vise à mettre en avant les forces, les faiblesses, les opportunités et les menaces du projet.  

%Table de l'analyse SWOT 
\begin{table}[H]
\caption{Analyse SWOT du projet}
\label{tb_swot}
% Please add the following required packages to your document preamble:

% If you use beamer only pass "xcolor=table" option, i.e. \documentclass[xcolor=table]{beamer}

\resizebox{\textwidth}{!}{%
\begin{tabular}{|c|l|l|}
\hline
 & \multicolumn{1}{c|}{\cellcolor[HTML]{92D050}\textbf{Forces}}       & \multicolumn{1}{c|}{\cellcolor[HTML]{ED7D31}\textbf{Faiblesses}} \\ \cline{2-3} 
\multirow{-2}{*}{\textbf{Interne}} &
  \begin{tabular}[c]{@{}l@{}}Donner accès aux informations relatives à l’université \\ Répond au besoin de guider des étudiants ou des visiteurs \\ Promotion des différents évènements du campus \\ Simple d’utilisation \\ Visible par tous à l’intérieur comme à l’extérieur du campus\end{tabular} &
  \begin{tabular}[c]{@{}l@{}}Trop grand nombre d’informations affichées \\ Illisibilité du contenu \\ Détournement de l’objectif premier (guider des \\ étudiants ou visiteurs) \\ Proportion entre bornes et étudiants/visiteurs\end{tabular} \\ \hline
 & \multicolumn{1}{c|}{\cellcolor[HTML]{00B0F0}\textbf{Opportunités}} & \multicolumn{1}{c|}{\cellcolor[HTML]{E73848}\textbf{Menaces}}    \\ \cline{2-3} 
\multirow{-2}{*}{\textbf{Externe}} &
  \begin{tabular}[c]{@{}l@{}}Possibilité d’extension sur les autres Campus de l’université \\ Partage d’information extérieur au Campus et à l’université \\ (évènements culturel) \\ Développement des supports digitaux \\ Augmente la visibilité des informations en dehors du campus\end{tabular} &
  \begin{tabular}[c]{@{}l@{}}Obsolescence face aux GSM \\ Manque d’intérêt des étudiants ou visiteurs \\ Dépendre du fournisseur en cas de problème \\ Plateformes (réseaux sociaux, site internet, etc.) \\ partageant les mêmes informations \\ Incivilités\end{tabular} \\ \hline
\end{tabular}%
}

\end{table}


\section{Forces}
\begin{itemize}

\item\textbf{Donner accès aux informations relatives à l’université}: Les bornes interactives contiendront des informations relatives à l’université telles que le plan détaillé du campus, les salles de cours, la localisation des différentes facultés, des bâtiments administratifs et les mettront ainsi à disposition de tous, que les personnes soient étudiantes ou de simples visiteurs. 

\item\textbf{Répondre au besoin de guider des étudiants ou des visiteurs}: Le but et la raison de la mise en place des bornes étant de guider des étudiants ou des visiteurs au sein du campus. Disposant d’un plan détaillé du campus à jour et disponible à tout moment dans la journée, les bornes répondent entière à ce besoin. 
\item\textbf{Promotions des différents évènements du campus}: Étant des bornes interactives, il sera possible d’interagir avec les bornes afin d’avoir l’information que nous souhaitons mais lorsqu’elles ne sont pas utilisées, elles deviendront alors des panneaux d’affichage et promouvront ainsi les différentes actualités et différents évènements qui auront lieu sur le campus. 
\item\textbf{Simple d'utilisation}: Les écrans tactiles étant complètement ancré dans notre quotidien il sera très facile d’utiliser une telle borne peu importe l’âge de l’utilisateur. 

\item\textbf{Visible par tous à l’intérieur comma à l’extérieur du campus}: Plusieurs bornes seront installées dans le campus mais aussi à l’extérieur du campus. Des personnes qui ne sont donc ni étudiant ni visiteurs pourront aussi avoir accès aux différentes informations qu’elles diffuseront. 

\end{itemize}
\begin{itemize}
\section{Faiblesses}

\item\textbf{Trop grand nombre d’informations affichées}: Vouloir afficher le plus d’informations possible peut entrainer une surcharge de l’information et par conséquent noyer les utilisateurs qui se retrouveront perdu. 

\item\textbf{Illisibilité du contenu}: Le trop plein d’informations présentes sur les bornes peut entrainer une difficulté de lecture et une confusion chez les utilisateurs mais va surtout desservir la borne en complexifiant son utilisation et provoquer une réticence ou encore un rejet des bornes par les utilisateurs. 
\item\textbf{Détournement de l’objectif de départ}: L’ajout de nouvelles fonctionnalités et informations peut faire oublier le but pour lequel les bornes furent installées et leur utilisation principale. 

\item\textbf{Proportion entre bornes et étudiants/visiteurs}: Le nombre de borne présent sur le campus sera insuffisant pour le nombre de personnes susceptibles de les utiliser. Il se peut que cela pousse les personnes à utiliser un autre support. 

\end{itemize}


\section{Opportunités}
\begin{itemize}
    \item \textbf{Possibilité d’extension sur les autres Campus de l’université}: Débutant le projet uniquement sur le campus de l’ULB, la possibilité d’implanter des bornes sur les autres campus de l’ULB voir même des universités partenaires de l’ULB afin d’augmenter la visibilité, la porter des informations. On ne parle plus du plan du campus mais des évènements qui s’y dérouleront pour y attirer le plus de monde. 
    \item \textbf{Partage d’information extérieur au Campus et à l’université (évènements culturel)} : La ville de Bruxelles ou d’autres organisations vont pouvoir aussi utiliser les bornes du campus pour y partager, présenter la multitude d’évènements qui ont lieu dans la ville. Les bornes permettent d’attirer, d’inciter directement les étudiants à participer aux évènements. 
    \item \textbf{Développement des supports digitaux}: Les nouvelles technologies permettent d’être connecté en permanence à un réseau et sont-elles mêmes connectées entre elles. La borne pourra donc échanger avec tout autre nouvelle technologie comme par exemple les smartphones afin d’y partager ou transférer n’importe quelles informations. 
    \item \textbf{Augmente la visibilité des informations en dehors du campus} : La possibilité que des gens n’ayant aucun rapport avec l’université aient accès aux informations du campus va favoriser le bouche-à-oreille et augmenter la visibilité et la porter de ces informations.
\end{itemize}

\section{Menaces}
\begin{itemize}
    \item \textbf{Obsolescence face aux GSM}: La totalité des étudiants et la quasi-totalité des visiteurs possèdent un GSM qui permet un accès rapide au plan du campus et rendre les bornes obsolètes.
    \item \textbf{Manque d’intérêt des étudiants ou visiteurs}: Les personnes ciblés peuvent se retrouver insensibles, non concernées par la mise en place et l’utilisation des bornes.
    \item \textbf{Dépendre du fournisseur en cas de problème} : L’éventualité d’un souci d’ordre numérique ou logiciel n’est également pas à exclure. Comme cela peut être observé dans certains lieux où ces bornes ont d’ores et déjà été adoptée, celles-ci ne sont pas pour autant infaillible. Il faudra donc faire appel au fournisseur et tant que les bornes ne sont pas réparées l’université est complètement dépendante de lui. 
    \item \textbf{Plateformes (réseaux sociaux, site internet, etc.) partageant les mêmes informations}: Les bornes auront pour concurrent direct les différentes plateformes en lignes et notamment les réseaux sociaux qui sont les premières sources d’informations des étudiants. Elles peuvent vite devenir un investissement inutile. 
    \item \textbf{Incivilités}: Il est possible que les bornes subissent des incivilités, des dégradations qui engendreront des coûts supplémentaires. 
\end{itemize}

 


 

 








 