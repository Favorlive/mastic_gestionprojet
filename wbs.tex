\chapter{Planning général  (WBS)}

\section{Work Breakdown Structure}
Le WBS ou Work Breakdown Structure est un découpage de projet en sous-ensembles ordonnés. Ce découpage va aboutir sur des sous-projets organisés en arborescence représentant des tâches à réaliser ou des livrables. L’objectif principal du WBS est de simplifier les projets en les réduisant plus facilement appréhendables. De plus, il facilite aussi l’organisation en permettant l’attribution des rôles et des responsabilités pour chaque branche. 

 

Il est primordial de définir les différentes taches à exécuter dans la réalisation d’un projet, le temps qu’elles requièrent et les acteurs qui les exécutent. Le découpage du WBS va permettre d’avoir une vision plus claire des étapes qui le composent, du temps que le projet prendra pour être aboutit. Cependant, il n’est pas possible de calculer de manière exacte la durée de chaque phase. Une estimation précise reste nécessaire afin d’éviter le plus possible de potentielles complications. 

Le WBS peut se retrouver sous deux formes différentes. Tout d’abord, la forme graphique représentant le projet découper sous forme d’arborescence. Ensuite, la forme textuelle du WBS quant à elle n’est qu’un simple tableau contenant les différentes phases et différents livrables du projet mais aussi la charge estimée des livrables. 
\begin{table}[H]
\resizebox{\textwidth}{!}{%
\begin{tabular}{|l|l|l|c|}
\hline
\rowcolor[HTML]{92D050} 
\multicolumn{1}{|c|}{\cellcolor[HTML]{92D050}\textbf{Planning}} &
  \multicolumn{1}{c|}{\cellcolor[HTML]{92D050}\textbf{Phases}} &
  \multicolumn{1}{c|}{\cellcolor[HTML]{92D050}\textbf{Tâches}} &
  \textbf{Charge estimée (j/h)} \\ \hline
\rowcolor[HTML]{E2EFD9} 
\multicolumn{1}{|c|}{\cellcolor[HTML]{92D050}} &
  \cellcolor[HTML]{C5E0B3} &
  \textbf{WP1.1 – Réunion de lancement} &
  \textbf{3 j/h} \\ \cline{3-4} 
\rowcolor[HTML]{E2EFD9} 
\multicolumn{1}{|c|}{\cellcolor[HTML]{92D050}} &
  \cellcolor[HTML]{C5E0B3} &
  \textbf{WP1.2 – État de la situation} &
  \textbf{15 j/h} \\ \cline{3-4} 
\rowcolor[HTML]{E2EFD9} 
\multicolumn{1}{|c|}{\cellcolor[HTML]{92D050}} &
  \cellcolor[HTML]{C5E0B3} &
  \textbf{WP1.3 – Définir les besoins} &
  \textbf{35 j/h} \\ \cline{3-4} 
\rowcolor[HTML]{E2EFD9} 
\multicolumn{1}{|c|}{\multirow{-4}{*}{\cellcolor[HTML]{92D050}\textbf{04/01/2021 – 14/04/2021}}} &
  \multirow{-4}{*}{\cellcolor[HTML]{C5E0B3}\textbf{WP1 - Plan d’action}} &
  \textbf{WP1.4 – Déterminer une solution} &
  \textbf{20 j/h} \\ \hline
\rowcolor[HTML]{E2EFD9} 
\cellcolor[HTML]{92D050} &
  \cellcolor[HTML]{C5E0B3} &
  \textbf{WP2.1 - Réunion de lancement} &
  \textbf{5 j/h} \\ \cline{3-4} 
\rowcolor[HTML]{E2EFD9} 
\cellcolor[HTML]{92D050} &
  \cellcolor[HTML]{C5E0B3} &
  \textbf{\begin{tabular}[c]{@{}l@{}}WP2.2 – Identifier les \\ utilisateurs cible\end{tabular}} &
  \textbf{10 j/h} \\ \cline{3-4} 
\rowcolor[HTML]{E2EFD9} 
\cellcolor[HTML]{92D050} &
  \cellcolor[HTML]{C5E0B3} &
  \textbf{WP2.3 – Budgétiser les actions} &
  \textbf{45 j/h} \\ \cline{3-4} 
\rowcolor[HTML]{E2EFD9} 
\multirow{-4}{*}{\cellcolor[HTML]{92D050}\textbf{15/03/2021 – 09/07/2021}} &
  \multirow{-4}{*}{\cellcolor[HTML]{C5E0B3}\textbf{WP2 - Plan Marketing}} &
  \textbf{WP2.4 – Appel d’offres} &
  \textbf{25 j/h} \\ \hline
\rowcolor[HTML]{E2EFD9} 
\cellcolor[HTML]{92D050} &
  \cellcolor[HTML]{C5E0B3} &
  \textbf{WP3.1 – Réunion de lancement} &
  \textbf{10 j/h} \\ \cline{3-4} 
\rowcolor[HTML]{E2EFD9} 
\cellcolor[HTML]{92D050} &
  \cellcolor[HTML]{C5E0B3} &
  \textbf{WP3.2 – Réalisation de la solution} &
  \textbf{90 j/h} \\ \cline{3-4} 
\rowcolor[HTML]{E2EFD9} 
\multirow{-3}{*}{\cellcolor[HTML]{92D050}\textbf{12/07/2021 – 10/12/2021}} &
  \multirow{-3}{*}{\cellcolor[HTML]{C5E0B3}\textbf{WP3 - Solution technique}} &
  \textbf{WP3.3 – Présentation de la solution} &
  \textbf{10 j/h} \\ \hline
\rowcolor[HTML]{E2EFD9} 
\cellcolor[HTML]{92D050} &
  \cellcolor[HTML]{C5E0B3} &
  \textbf{WP4.1 – Réunion de lancement} &
  \textbf{20 j/h} \\ \cline{3-4} 
\rowcolor[HTML]{E2EFD9} 
\cellcolor[HTML]{92D050} &
  \cellcolor[HTML]{C5E0B3} &
  \textbf{\begin{tabular}[c]{@{}l@{}}WP4.2 – Stratégie de \\ communication\end{tabular}} &
  \textbf{45 j/h} \\ \cline{3-4} 
\rowcolor[HTML]{E2EFD9} 
\multirow{-3}{*}{\cellcolor[HTML]{92D050}\textbf{26/07/2021 – 25/02/2022}} &
  \multirow{-3}{*}{\cellcolor[HTML]{C5E0B3}\textbf{WP4 - Plan de communication}} &
  \textbf{\begin{tabular}[c]{@{}l@{}}WP4.3 – Mise en place \\ de la communication\end{tabular}} &
  \textbf{90 j/h} \\ \hline
\rowcolor[HTML]{E2EFD9} 
\cellcolor[HTML]{92D050} &
  \cellcolor[HTML]{C5E0B3} &
  \textbf{WP5.1 – Réunion de lancement} &
  \textbf{10 j/h} \\ \cline{3-4} 
\rowcolor[HTML]{E2EFD9} 
\multirow{-2}{*}{\cellcolor[HTML]{92D050}\textbf{11/01/2021 – 29/04/2022}} &
  \multirow{-2}{*}{\cellcolor[HTML]{C5E0B3}\textbf{WP5 - Suivi}} &
  \textbf{WP5.2 – Rapport sur la situation} &
  \textbf{\begin{tabular}[c]{@{}c@{}}1 j/h  \\ Chaque semaine \\ pour chaque \\ domaine d’action \\ (technique, marketing, \\ communication, etc.)\end{tabular}} \\ \hline
\rowcolor[HTML]{E2EFD9} 
\cellcolor[HTML]{92D050}\textbf{4/01/2021 - 29/04/2022} &
  \cellcolor[HTML]{C5E0B3}\textbf{Projet global} &
  \textbf{WP1 – Réalisation globale du projet} &
  \textbf{350 j/h} \\ \hline
\end{tabular}%
}
\end{table}


Le WBS a permit d’estimer la charge pour toutes les tâches de notre projet. Cependant, il ne donne pas la possibilité de voir l’étalement des tâches dans le temps. C’est alors qu’intervient le diagramme de GANTT. 

\section{Diagramme de GANTT}
Ce diagramme va aider à présenter chronologiquement ainsi que représenter visuellement l'état d'avancement des différentes tâches qui constituent un projet. 

 

L'intérêt de ce diagramme est qu'il est possible d'y représenter les dépendances entre les tâches, le degré d'accomplissement de chaque tâche à tout moment, les ressources impliquées et ainsi de suite. 