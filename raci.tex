\chapter{Gouvernance du projet (modèle RACI)}

\section{Rôles et tâches}
La réussite du projet dépend en grande partie de la répartition des rôles et des tâches de chacun.  Il est capital de déterminer une équipe projet soumise à l’autorité d’un chef de projet. La matrice RACI définit ainsi le champ d’action de chacun et un cadre formel. 

Cette approche fixe quatre rôles. Tout d’abord, la ou les personnes qui réalisent l’action sont représentées par un R (responsible).  Ensuite, la personne qui supervise est représenté par un A (accountable) supervise celle qui réalise l’action. De plus, la ou les personnes qui vont être consultées dans la réalisation de la tâche sont représentées par un C (consulted). Enfin, la ou les personnes qui doivent être informées sont représentées par un I (informed). 




\section{Intervenants}
En ce qui concerne les intervenants, il a été décidé de ne choisir que cinq catégories d’intervenants, qui sont les suivantes:

\begin{itemize}
    \item \textbf{Bureau exécutif}: Composé du président de l’université ainsi que son vice-président. On pourrait aussi y intégrer le recteur et les représentant des étudiants. 

    \item \textbf{Trésorier}: On y retrouve tout simplement le trésorier et le trésorier adjoint de l’université. Ils sont chargés du budget de l’université et par conséquent du budget octroyé au projet. 

    \item \textbf{Chef de projet}: Les personnes responsables de la gestion globale du projet seront le chef du service communication et le chef du service informatique de l’ULB. 

    \item \textbf{Équipe du projet}: L’équipe qui sera chargée de réaliser,mettre en place techniquement le projet sera composé de membres du service information et communication de l’université ainsi que le fournisseur des bornes interactives. 

    \item \textbf{Utilisateurs cibles}: Les utilisateurs ciblent qui seront les principaux concernés par le projet sont composés des étudiants et des visiteurs du campus du SOLBOSCH.      
    
\end{itemize}

Il a été décidé de mettre en place deux personnes en tant que chef de projet puisque le projet est à vocation communicationnelle, mais il y a une grande partie technique. Le responsable de service communication s’occupera des informations qui seront affichées sur les bornes interactives. Tandis que le responsable informatique, supervisera la réalisation technique du projet. 


Dans l’équipe de projet, on note la présence du fournisseur des bornes, interactives. Il sera chargé de fournir les bornes mais aussi de former les personnes de l’université quant à leur utilisation, mais ne peut être chef de projet étant donné qu’il n’aura aucun impact décisionnel sur la finalité du projet. 

\begin{table}[H]

\resizebox{\textwidth}{!}{%
\begin{tabular}{|
>{\columncolor[HTML]{ED7D31}}c |c|c|c|c|c|}
\hline
\multicolumn{1}{|l|}{\cellcolor[HTML]{ED7D31}\textbf{Tâches / Intervenants}} &
  \cellcolor[HTML]{ED7D31}\textbf{\begin{tabular}[c]{@{}c@{}}Bureau \\ exécutif\end{tabular}} &
  \cellcolor[HTML]{ED7D31}\textbf{Trésorier} &
  \cellcolor[HTML]{ED7D31}\textbf{\begin{tabular}[c]{@{}c@{}}Chef de \\ projet\end{tabular}} &
  \cellcolor[HTML]{ED7D31}\textbf{\begin{tabular}[c]{@{}c@{}}Équipe de \\ projet\end{tabular}} &
  \cellcolor[HTML]{ED7D31}\textbf{\begin{tabular}[c]{@{}c@{}}Utilisateurs\\  (étudiants/\\ visiteurs)\end{tabular}} \\ \hline
\textbf{\begin{tabular}[c]{@{}c@{}}Définition du \\ périmètre\end{tabular}}                 & \textbf{A} & \textbf{C} & \textbf{R} & \textbf{C} & \textbf{I} \\ \hline
\textbf{\begin{tabular}[c]{@{}c@{}}Préparer le \\ Business Case\end{tabular}}               & \textbf{C} & \textbf{A} & \textbf{R} & \textbf{R} & \textbf{C} \\ \hline
\textbf{\begin{tabular}[c]{@{}c@{}}Préparer le planning \\ du projet\end{tabular}}          & \textbf{C} & \textbf{A} & \textbf{R} & \textbf{R} & \textbf{I} \\ \hline
\textbf{Gestion des risques}                                                                & \textbf{A} & \textbf{C} & \textbf{R} & \textbf{R} & \textbf{C} \\ \hline
\textbf{\begin{tabular}[c]{@{}c@{}}Analyse besoins \\ métiers/fonctionnels\end{tabular}}    & \textbf{C} & \textbf{A} & \textbf{R} & \textbf{R} & \textbf{C} \\ \hline
\textbf{\begin{tabular}[c]{@{}c@{}}Déterminer le \\ lancement du projet\end{tabular}}       & \textbf{A} & \textbf{C} & \textbf{I} & \textbf{I} & \textbf{C} \\ \hline
\textbf{\begin{tabular}[c]{@{}c@{}}Phases de test \\ (réalisation/validation)\end{tabular}} & \textbf{I} & \textbf{I} & \textbf{A} & \textbf{R} & \textbf{C} \\ \hline
\textbf{\begin{tabular}[c]{@{}c@{}}Collecter les \\ retours utilisateurs\end{tabular}}      & \textbf{I} & \textbf{I} & \textbf{A} & \textbf{R} & \textbf{C} \\ \hline
\end{tabular}%
}

\end{table}

\section{Descriptions rôles au sein des tâches }
    \subsection{Définition du périmètre}
    Le bureau exécutif aura pour rôle de valider le périmètre du projet afin de le valider et ainsi acter le début de la réalisation de celui-ci. Ce périmètre sera effectué par le chef de projet en consultation avec bien évidemment son équipe mais aussi le gestionnaire du budget en l’occurrence le trésorier, mais il sera celui qui prendra la décision finale. Les utilisateurs quant à eux seront juste informés du périmètre convenu étant donné qu’ils en seront les premiers impactés. 

\subsection{Préparer le Business Case }
    Le bureau exécutif sera ici simple consultant dans la réalisation du business case tout comme les utilisateurs. C’est le trésorier qui sera responsable de l’aspect final du business case et devra rendre des comptes tout au long de la préparation du business case. Le chef du projet ainsi que l’équipe du projet seront chargés de l’accomplissement de cette tâche. 

 
\subsection{Préparer le planning du projet}
    Tout comme pour la préparation du Business Case, les rôles seront quasiment identiques. Seuls les utilisateurs passeront de leur rôle de consultant, au simple fait qu’ils seront informés durant la confection du planning du projet. 

 
\subsection{Gestion des risques}
    Concernant la gestion des risques, le bureau exécutif aura le rôle de responsable, puisqu’il pourra décider si le projet est maintenu ou non en fonction des risques découvert et surtout la manière dont ils seront gérés. Le chef de projet et son équipe comme pour la totalité du projet seront à la réalisation de la tâche et se chargeront de gérer les risques. Le trésorier sera consultant, surtout pour les risques en lien avec le budget et les utilisateurs ayant le même rôle que le trésorier mais cela concernera l’aspect technique du projet. 

 
\subsection{Analyse des besoins métiers/fonctionnels}
    Le bureau exécutif pour cette tâche redevient un consultant, au même point que les utilisateurs. Ils vont pouvoir donner leurs avis concernant les besoins qui devront être priorisés ou pas. Le trésorier sera la personne en charge de cette tâche de valider la décision finale mais la réalisation revient encore au chef de projet et à l’équipe de projet. 

 
\subsection{Déterminer le lancerment du projet}
    Le bureau exécutif aura pour rôle de déterminer une date pour le lancement de projet. Il pourra égaler réévaluer cette date tout au long du projet. Le trésorier et les utilisateurs seront dans ce cas présent consulté dans la prise de décision, surtout concernant les utilisateurs étant donné qu’ils sont les cibles du projet. Enfin le chef de projet et son équipe seront simplement informé de la date du lancement et elle constituera leur deadline pour la réalisation du projet. 

 
\subsection{Phases de tests (réalisation/validation)}
    Le chef de projet est la personne ne charge de cette tâche et c’est par conséquent l’équipe du projet qui aura pour mission de la réaliser. Les utilisateurs seront consultants puisque les tests seront en parti effectué sur eux. Enfin le bureau exécutif et le trésorier seront informés des résultats obtenus à la fin de cette tâche. 

 
\subsection{Collecter les retours utilisateurs}
    Comme pour la phase de test, le chef de projet sera encore en charge de la tâche et son équipe se contentera de la réaliser. Les utilisateurs seront de nouveau consultants puisqu’il s’agit ici de récupérer leurs impressions. Elles seront directement transmises au bureau exécutif et au trésorier qui ont besoin de connaître ces informations. 